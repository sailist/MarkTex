\documentclass{article}%
\usepackage{lastpage}%
\usepackage{ragged2e}%
\usepackage{pifont}%
\usepackage{graphicx}%
\usepackage{amsmath}%
%
\usepackage{amsmath}%提供数学公式支持

\usepackage{graphics}%用于添加图片
\usepackage{graphicx}%加强插图命令
\newcommand{\figpath}[1]{contents/fig/#1}

\usepackage{fontspec}%用于配置字体
\usepackage[table]{xcolor}%用于各种颜色环境
\usepackage{enumitem}%用于定制list和enum
\usepackage{float}%用于控制Float环境,添加H参数(强制放在Here)
\usepackage[colorlinks,linkcolor=airforceblue,urlcolor=blue,anchorcolor=blue,citecolor=green]{hyperref}%用于超链接,另外添加该包目录会自动添加引用。

\usepackage[most]{tcolorbox}%用于添加各种边框支持
\usepackage[cache=true,outputdir=./out]{minted}%如果不保留临时文件就设置cache=false,如果输出设置了其他目录那么outputdir参数也有手动指定,否则会报错。
\tcbuselibrary{minted}%加载tcolorbox的代码风格

\usepackage[a4paper,left=3cm,right=3cm,top=3cm,bottom=3cm]{geometry}%用于控制版式
\usepackage{appendix}%用于控制附加文件
\usepackage{ifthen}

\usepackage{pdfpages}%用于支持插入其他pdf页
\usepackage{booktabs}%目前用于给表格添加 \toprule \midrule 等命令
\usepackage{marginnote} %用于边注
\usepackage[pagestyles,toctitles]{titlesec} %用于标题格式DIY
% \usepackage{fancyhdr}%用于排版页眉页脚
\usepackage{ragged2e} % 用于对齐
%\usepackage{fixltx2e} %用于文本环境的下标 % 2015 版本后已经不再需要了
\usepackage{ulem} %用于好看的下划线、波浪线等修饰
\usepackage{pifont} %数学符号
\usepackage{amssymb} %数学符号

%\usepackage{fontspec}
%\setmainfont{DejaVu Serif}


\definecolor{langback}{RGB}{245,244,250}
\definecolor{langbacktitle}{RGB}{235,233,245}
\definecolor{langtitle}{RGB}{177,177,177}
\definecolor{langno}{RGB}{202,202,202}
\tcbset{arc=1mm}
\renewcommand{\theFancyVerbLine}{\sffamily\textcolor{langno}{\scriptsize\oldstylenums{\arabic{FancyVerbLine}}}}%重定义行号的格式
\newtcblisting{langbox}[1][tex]{%参考自https://reishin.me/tmux/ 的代码框样式
    arc=1mm,breakable,
    colframe=langbacktitle,
    colbacktitle=langbacktitle,
    coltitle=langtitle,
    fonttitle=\bfseries\sffamily,
    lefttitle=1mm,toptitle=0.5mm,bottomtitle=0.5mm,
    title = Code,
    drop shadow,
    listing engine=minted,
    minted style=colorful,
    minted language=#1,
    minted options={fontsize=\small,breaklines,autogobble,linenos,numbersep=2mm,xleftmargin=1mm},
    colback=langback,listing only,
    bottomrule=0mm,leftrule=0mm,toprule=0mm,rightrule=0mm,
    enhanced,
    % overlay={\begin{tcbclipinterior}\fill[langback] (frame.south west)rectangle ([xshift=5mm]frame.north west);\end{tcbclipinterior}}
}

\definecolor{boxback}{RGB}{245,246,250}
\newtcolorbox{markquote}{
    colback=boxback,fonttitle=\sffamily\bfseries,arc=0pt,breakable,
    boxrule=0pt,bottomrule=-1pt,toprule=-1pt,leftrule=-1pt,rightrule=-1pt,
    drop shadow,enhanced
}

\usepackage[UTF8,fontset=windowsnew,heading=true]{ctex}
\ctexset{
	section = {
	number = 第\chinese{section}章,
	format = \zihao{3}\bfseries,
	},
	subsection = {
	number = \arabic{section}.\arabic{subsection},
	format = \Large\bfseries
	},
	subsubsection = {
	number = \arabic{section}.\arabic{subsection}.\arabic{subsubsection},
	format = \Large\bfseries,
	},
    paragraph = {
	format = \large\bfseries,
	},
    subparagraph = {
	format = \large\bfseries,
	},
}

\setlength{\parindent}{2em}%设置首行缩进
\linespread{1.3}%设置行距

\setlength{\parskip}{0.5em}%设置段间距
\setcounter{tocdepth}{4}%设置目录级数
\setcounter{secnumdepth}{3}


\newtcbox{\inlang}[1][red]{on line,
arc=0pt,outer arc=0pt,colback=#1!10!white,colframe=#1!50!black,
boxsep=0pt,left=1pt,right=1pt,top=2pt,bottom=2pt,
boxrule=0pt,bottomrule=-1pt,toprule=-1pt,leftrule=-1pt,rightrule=-1pt}

\newlength\tablewidth


\definecolor{tablelinegray}{RGB}{221,221,221}
\definecolor{tablerowgray}{RGB}{247,247,247}
\definecolor{tabletopgray}{RGB}{245,246,250}
\definecolor{airforceblue}{rgb}{0.36, 0.54, 0.66}

%
\title{MarkTex特性说明}%
\author{sailist}%
%
\begin{document}%
\normalsize%
\maketitle%
%
\tableofcontents
\newpage%

%
%
%

%

%
本文用于演示和测试转换后的效果%
\section{基本特性\textsubscript{下标在这里}}%

%

%
\begin{itemize}%
\item%
支持目前主流的所有markdown语法,包括脚注、xml%
\item%
额外添加了下划线语法(\inlang{\small{\_\_下划线\_\_}})%
\item%
表格自动调整列宽%
\item%
复选框支持三种%
\item%
无论是本地图片还是网络图片,都能够支持。%
\end{itemize}%
%

%

%
\section{各类文本和标题级别}%

%
支持一般的文本和\textbf{加粗},\textit{斜体},\inlang{\small{行内代码}},和 $InLine Formula$ ,\href{http://github.com}{超链接}。%

%
同时,支持多级嵌套,包括\textbf{\textit{粗斜体}},\textbf{\underline{粗体+下划线}},\textit{\underline{斜体+下划线}}等等,\textbf{\textit{\underline{粗斜体+下划线}}}%

%
\sout{删除线},\underline{下划线}%

%
\subsection{二级标题}%

%

%
\subsubsection{三级标题}%

%
目录编号支持到三级标题,可以通过修改latex文件或者直接更改模板来完成。%

%
\paragraph{四级标题}%

%
\subparagraph{五级标题}%

%

%
\section{脚注}%

%

%
可以支持脚注格式\footnote{这里是脚注的内容,新版支持在脚注中的部份字体,包括 \textbf{加粗},\textit{斜体}等}%

%
%

%

%
\section{表格}%

%
支持一般的文本格式,暂时不支持表格内图片。另外,表格取消了浮动(float),因此不支持对表格的描述(caption),不过在Markdown中也没有对表格的描述,因此也不算功能不完善。%

%
\begin{center}%
\setlength\tablewidth{\dimexpr (\textwidth -4\tabcolsep)}%
\arrayrulecolor{tablelinegray!75}%
\rowcolors{2}{tablerowgray}{white}%
\begin{tabular}{|p{0.500\tablewidth}<{\centering}|p{0.500\tablewidth}<{\centering}|}%
\hline%
\rowcolor{tabletopgray}%
\textbf{ColA}&\textbf{ ColB }\\%
\hline%
 \textbf{Table Bold} &  \textit{Table Italic}\\%
\hline%
 \inlang{\small{Table Code}} &   $Table Formula$ \\%
\hline%
\href{http:///www.github.com}{Table line}&Table Text\\%
\hline%
\end{tabular}%
\end{center}%
%

%
\begin{center}%
\setlength\tablewidth{\dimexpr (\textwidth -8\tabcolsep)}%
\arrayrulecolor{tablelinegray!75}%
\rowcolors{2}{tablerowgray}{white}%
\begin{tabular}{|p{0.077\tablewidth}<{\centering}|p{0.077\tablewidth}<{\centering}|p{0.077\tablewidth}<{\centering}|p{0.769\tablewidth}<{\centering}|}%
\hline%
\rowcolor{tabletopgray}%
\textbf{A}&\textbf{B}&\textbf{C}&\textbf{Long Text Sample Long Text Sample Long Text Sample Long Text Sample Long Text Sample Long Text Sample }\\%
\hline%
A&B&C&D\\%
\hline%
A&B&C&D\\%
\hline%
A&B&C&D\\%
\hline%
\end{tabular}%
\end{center}%
%

%
\section{列表和序号/itemize\&enumerate}%

%

%
支持无序号列表,序号列表,复选框%

%
\begin{itemize}%
\item%
支持\textbf{加粗},\textit{斜体},\inlang{\small{行内代码}}, $Inline Formula$ ,\href{http:///www.github.com}{超链接}%
\item%
支持\textbf{加粗},\textit{斜体},\inlang{\small{行内代码}}, $Inline Formula$ ,\href{http:///www.github.com}{超链接}%
\item%
支持\textbf{加粗},\textit{斜体},\inlang{\small{行内代码}}, $Inline Formula$ ,\href{http:///www.github.com}{超链接}%
\end{itemize}%
%

%
\begin{enumerate}%
\item%
支持\textbf{加粗},\textit{斜体},\inlang{\small{行内代码}}, $Inline Formula$ ,\href{http:///www.github.com}{超链接}%
\item%
支持\textbf{加粗},\textit{斜体},\inlang{\small{行内代码}}, $Inline Formula$ ,\href{http:///www.github.com}{超链接}%
\item%
支持\textbf{加粗},\textit{斜体},\inlang{\small{行内代码}}, $Inline Formula$ ,\href{http:///www.github.com}{超链接}%
\end{enumerate}%
%

%
\begin{itemize}%
\item[\rlap{\raisebox{0.3ex}{\hspace{0.4ex}\scriptsize \ding{56}}}$\square$]%
支持%
\item[$\square$]%
三种%
\item[\rlap{\raisebox{0.3ex}{\hspace{0.4ex}\tiny \ding{52}}}$\square$]%
复选框格式%
\item[$\square$]%
复选框格式%
\end{itemize}%
%

%
\section{图片}%

%
支持网络图片,会在转换的时候自动下载到本地,同时对非 JPG/PNG 格式的图片,会将其转换为PNG格式。所有的图片会被hash 后放置在 cacheimg\_dir 下,默认该目录为 <output\_dir>/imgs%

%
\subsection{行内图片}%

%

%
最新版本添加了行内图片,如果没有换行,那么该图片会被人为是行内图片,会自动调整高度适应一行:\raisebox{-0.5mm}{\includegraphics[height=1em]{imgs/1c59f8ef2aa3c5e527a22b7c258489d6.png}}%

%
测试2:\raisebox{-0.5mm}{\includegraphics[height=1em]{imgs/23a8048a54a8f32cd7c0573a1e52f239.png}}图片之后%

%
\subsection{行间图片}%

%

%
\begin{center}%


\begin{figure}[H]%
%
\includegraphics[width=0.8\textwidth]{imgs/1c59f8ef2aa3c5e527a22b7c258489d6.png}%
\caption{可以添加图片描述}%
\end{figure}

%
\end{center}%

%
相对路径:%
\begin{center}%


\begin{figure}[H]%
%
\includegraphics[width=0.8\textwidth]{imgs/38c1cb2db4236befd1b8075f1cf21e34.png}%
\end{figure}

%
\end{center}%

%
\section{公式}%

%
公式支持中文,但没有编号,如果要编号可以通过手动添加tag的方式%

%
\subsection{行内公式}%

%
 $f(x) = x\_\{1\} \text{中文}$ %

%
\subsection{行间公式}%

%
\[%
\\ \text{使用函数} f(x_i)=ax_i+b \tag{1}\\ \\%
\]%
%

%

%

%
\subsection{符号支持}%

%
符号集在内部做了一个映射,可以将任意公式内外的符号均映射成为 LaTeX 中的符号。%

%
原本的解决方案为添加一个额外的符号字体集来解决(来自于\href{https://tex.stackexchange.com/questions/69901/how-to-typeset-greek-letters}{stackoverflow}),目前的方案为两者优先采用映射方法,目前支持的符号列举如下(可能支持更多符号,但没有经过测试):%

%
\subsubsection{希腊字母}%

%
$\alpha{}$$\beta{}$$\gamma{}$$\delta{}$$\varepsilon{}$$\zeta{}$$\eta{}$$\theta{}$$\imath{}$$\kappa{}$$\lambda{}$$\mu{}$$\vartheta{}$$\xi{}$$o{}$$\pi{}$$\rho{}$$\sigma{}$$\tau{}$$\upsilon{}$$\phi{}$$\chi{}$$\psi{}$$\omega{}$%

%
$A{}$$B{}$$\Gamma{}$$\triangle{}$$E{}$$Z{}$$H{}$$\Theta{}$$I{}$$K{}$$\wedge{}$$M{}$$N{}$$\Xi{}$$O{}$$\sqcap{}$$P{}$$\sum{}$$T{}$$Y{}$$\Phi{}$$X{}$$\Psi{}$$\Omega{}$%

%
 $\alpha{}\beta{}\gamma{}\delta{}\varepsilon{}\zeta{}\eta{}\theta{}\imath{}\kappa{}\lambda{}\mu{}\vartheta{}\xi{}o{}\pi{}\rho{}\sigma{}\tau{}\upsilon{}\phi{}\chi{}\psi{}\omega{}$ %

%
\[%
\alpha{}\beta{}\gamma{}\delta{}\varepsilon{}\zeta{}\eta{}\theta{}\imath{}\kappa{}\lambda{}\mu{}\vartheta{}\xi{}o{}\pi{}\rho{}\sigma{}\tau{}\upsilon{}\phi{}\chi{}\psi{}\omega{}\\%
\]%
%
\subsubsection{运算符号}%

%
$\pm{}$$\times{}$$\div{}$$|{}$$\not{|}{}$%

%
$\cdot{}$$unknown$$\ast{}$$\odot{}$$\oplus{}$%

%
$\leq{}$$\geq{}$$\neq{}$$\approx{}$$\equiv{}$%

%
$\sum{}$$\sqcap{}$$\amalg{}$$\in{}$$\notin{}$$\subset{}$$\supset{}$$\subseteq{}$$\supseteq{}$$\subset{}$$unknown$%

%
$\wedge{}$$\vee{}$$\cap{}$$\cup{}$$\exists{}$$\forall{}$$\triangledown{}$%

%
$\bot{}$$\angle{}$%

%
$\infty{}$$unknown$$'{}$%

%
$\int{}$$\iint{}$$\iiint{}$%

%
$\uparrow{}$$\downarrow{}$$\leftarrow{}$$\to{}$$\leftrightarrow{}$$\updownarrow{}$%

%
\section{代码}%

%
代码使用tcolorbox和minted,基本支持所有主流语言。支持的所有语言请参考 \href{https://www.overleaf.com/learn/latex/Code_Highlighting_with_minted}{Code Highlighting with minted} ,因此在添加代码环境的时候请注意标注在```后的代码语言和minted支持的相同,其中一部分minted和markdown中标识不相同的语言都做了映射(如markdown中是cpp但minted中是c++,以及javascrip和js),如果仍然存在转换错误,请手动调整语言类型或者提交错误给我由我来更新项目。%

%
\begin{langbox}[Python]%
if __name__ == "__main__":%
print("hello world!")%
\end{langbox}%
%

%
\begin{langbox}[C++]%
#include<stdio.h>%
int main(){%
printf("hello world")%
return 0;%
}%
%
\end{langbox}%
%

%
\section{引用}%

%
\begin{markquote}%
引用内环境和普通文本基本一致,但是不支持标题,不支持代码。%

%
不支持代码。由于LaTeX中环境嵌套导致过长的代码使得pdf无法换页,因此我取消了在引用中行间代码的支持,在引用中检测到代码环境会从引用环境中跳出跳出。%

%
演示\textbf{加粗},\textit{斜体},\inlang{\small{行内代码}}, $Inline Formula$ ,\href{http:///www.github.com}{超链接}%

%
\begin{itemize}%
\item%
支持\textbf{加粗},\textit{斜体},\inlang{\small{行内代码}}, $Inline Formula$ ,\href{http:///www.github.com}{超链接}%
\end{itemize}%

%
\begin{enumerate}%
\item%
支持\textbf{加粗},\textit{斜体},\inlang{\small{行内代码}}, $Inline Formula$ ,\href{http:///www.github.com}{超链接}%
\end{enumerate}%

%
\begin{markquote}%

%

%
新版 MarkTex 终于支持多级嵌套引用了!%

%
\begin{markquote}%
\begin{markquote}%
\begin{markquote}%
\begin{markquote}%
 $ f(x) = ax+b$ %

%
\end{markquote}%

%
\end{markquote}%

%
\end{markquote}%

%
\end{markquote}%

%
任意级别的嵌套完全没有问题!%

%
\end{markquote}%

%

%

%
\end{markquote}%
%

%
\begin{markquote}%
表格:%

%
\begin{center}%
\setlength\tablewidth{\dimexpr (\textwidth -4\tabcolsep)}%
\arrayrulecolor{tablelinegray!75}%
\rowcolors{2}{tablerowgray}{white}%
\begin{tabular}{|p{0.500\tablewidth}<{\centering}|p{0.500\tablewidth}<{\centering}|}%
\hline%
\rowcolor{tabletopgray}%
\textbf{ColA}&\textbf{ ColB }\\%
\hline%
 \textbf{Table Bold} &  \textit{Table Italic}\\%
\hline%
 \inlang{\small{Table Code}} &   $Table Formula$ \\%
\hline%
\href{http:///www.github.com}{Table line}&Table Text\\%
\hline%
\end{tabular}%
\end{center}%

%
公式:%

%
\[%
F(x_i) = wx_i+b\\%
\]%

%
图片由于引用环境的问题,不支持浮动窗口,因此无法添加描述,描述会被忽略。%

%
\begin{center}%
\includegraphics[width=0.8\textwidth]{imgs/1c59f8ef2aa3c5e527a22b7c258489d6.png}%
\end{center}%

%

%

%
\end{markquote}%
%

%

%

%
\section{新特性{-}include}%

%

%
非常酷的特性!可以使用特殊的html标签来引入其他的 MarkDown 或者 LaTeX 文件!%
\subsection{引入 Markdown 文档}%

%

%
%
%

%
\section{表格}%

%
支持一般的文本格式,暂时不支持表格内图片。另外,表格取消了浮动(float),因此不支持对表格的描述(caption),不过在Markdown中也没有对表格的描述,因此也不算功能不完善。%

%

%
\begin{center}%
\setlength\tablewidth{\dimexpr (\textwidth -4\tabcolsep)}%
\arrayrulecolor{tablelinegray!75}%
\rowcolors{2}{tablerowgray}{white}%
\begin{tabular}{|p{0.500\tablewidth}<{\centering}|p{0.500\tablewidth}<{\centering}|}%
\hline%
\rowcolor{tabletopgray}%
\textbf{ColA}&\textbf{ ColB }\\%
\hline%
 \textbf{Table Bold} &  \textit{Table Italic}\\%
\hline%
 \inlang{\small{Table Code}} &   $Table Formula$ \\%
\hline%
\href{http:///www.github.com}{Table line}&Table Text\\%
\hline%
\end{tabular}%
\end{center}%

%
%

%

%
\begin{center}%
\setlength\tablewidth{\dimexpr (\textwidth -8\tabcolsep)}%
\arrayrulecolor{tablelinegray!75}%
\rowcolors{2}{tablerowgray}{white}%
\begin{tabular}{|p{0.077\tablewidth}<{\centering}|p{0.077\tablewidth}<{\centering}|p{0.077\tablewidth}<{\centering}|p{0.769\tablewidth}<{\centering}|}%
\hline%
\rowcolor{tabletopgray}%
\textbf{A}&\textbf{B}&\textbf{C}&\textbf{Long Text Sample Long Text Sample Long Text Sample Long Text Sample Long Text Sample Long Text Sample }\\%
\hline%
A&B&C&D\\%
\hline%
A&B&C&D\\%
\hline%
A&B&C&D\\%
\hline%
\end{tabular}%
\end{center}%

%
%

%
%
%

%
\section{公式}%

%
公式支持中文,但没有编号,如果要编号可以通过手动添加tag的方式%

%
\subsection{行内公式}%

%
 $f(x) = x\_\{1\} \text{中文}$ %

%
\subsection{行间公式}%

%

%
\[%
\\ \text{使用函数} f(x_i)=ax_i+b \tag{1}\\ \\%
\]%

%
%

%

%

%
\subsection{符号支持}%

%
符号集在内部做了一个映射,可以将任意公式内外的符号均映射成为 LaTeX 中的符号。%

%
原本的解决方案为添加一个额外的符号字体集来解决(来自于\href{https://tex.stackexchange.com/questions/69901/how-to-typeset-greek-letters}{stackoverflow} ),目前的方案为两者优先采用映射方法,目前支持的符号列举如下(可能支持更多符号,但没有经过测试):%

%
\subsubsection{希腊字母}%

%
$\alpha{}$$\beta{}$$\gamma{}$$\delta{}$$\varepsilon{}$$\zeta{}$$\eta{}$$\theta{}$$\imath{}$$\kappa{}$$\lambda{}$$\mu{}$$\vartheta{}$$\xi{}$$o{}$$\pi{}$$\rho{}$$\sigma{}$$\tau{}$$\upsilon{}$$\phi{}$$\chi{}$$\psi{}$$\omega{}$%

%
$A{}$$B{}$$\Gamma{}$$\triangle{}$$E{}$$Z{}$$H{}$$\Theta{}$$I{}$$K{}$$\wedge{}$$M{}$$N{}$$\Xi{}$$O{}$$\sqcap{}$$P{}$$\sum{}$$T{}$$Y{}$$\Phi{}$$X{}$$\Psi{}$$\Omega{}$%

%

%
\textbf{$\alpha{}$$\beta{}$$\gamma{}$$\delta{}$$\varepsilon{}$$\zeta{}$$\eta{}$$\theta{}$$\imath{}$$\kappa{}$$\lambda{}$$\mu{}$$\vartheta{}$$\xi{}$$o{}$$\pi{}$$\rho{}$$\sigma{}$$\tau{}$$\upsilon{}$$\phi{}$$\chi{}$$\psi{}$$\omega{}$}%

%
\inlang{\small{$\alpha{}$$\beta{}$$\gamma{}$$\delta{}$$\varepsilon{}$$\zeta{}$$\eta{}$$\theta{}$$\imath{}$$\kappa{}$$\lambda{}$$\mu{}$$\vartheta{}$$\xi{}$$o{}$$\pi{}$$\rho{}$$\sigma{}$$\tau{}$$\upsilon{}$$\phi{}$$\chi{}$$\psi{}$$\omega{}$}}%

%
 $\alpha{}\beta{}\gamma{}\delta{}\varepsilon{}\zeta{}\eta{}\theta{}\imath{}\kappa{}\lambda{}\mu{}\vartheta{}\xi{}o{}\pi{}\rho{}\sigma{}\tau{}\upsilon{}\phi{}\chi{}\psi{}\omega{}$ %

%

%
\[%
\alpha{}\beta{}\gamma{}\delta{}\varepsilon{}\zeta{}\eta{}\theta{}\imath{}\kappa{}\lambda{}\mu{}\vartheta{}\xi{}o{}\pi{}\rho{}\sigma{}\tau{}\upsilon{}\phi{}\chi{}\psi{}\omega{}\\%
\]%

%
%

%

%
\begin{langbox}[Tex]%
ΑΒΓΔΕΖΗΘΙΚΛΜΝΞΟΠΡΣΤΥΦΧΨΩ%
\end{langbox}%

%
%

%
\subsubsection{运算符号}%

%
$\pm{}$$\times{}$$\div{}$$|{}$$\not{|}{}$%

%
$\cdot{}$$unknown$$\ast{}$$\odot{}$$\oplus{}$%

%
$\leq{}$$\geq{}$$\neq{}$$\approx{}$$\equiv{}$%

%
$\sum{}$$\sqcap{}$$\amalg{}$$\in{}$$\notin{}$$\subset{}$$\supset{}$$\subseteq{}$$\supseteq{}$$\subset{}$$unknown$%

%
$\wedge{}$$\vee{}$$\cap{}$$\cup{}$$\exists{}$$\forall{}$$\triangledown{}$%

%
$\bot{}$$\angle{}$%

%
$\infty{}$$unknown$$'{}$%

%
$\int{}$$\iint{}$$\iiint{}$%

%
$\uparrow{}$$\downarrow{}$$\leftarrow{}$$\to{}$$\leftrightarrow{}$$\updownarrow{}$

%

%
%

%
\subsection{引入 LaTeX 文档}%

%
%

%

%
可以通过使用include标签插入tex原文件,不过注意,不需要添加文档区,文档类等声明,插入的方式为完全将原文件复制粘贴到相应位置。
%

%

%

%
也因此,这种方式下如果使用了额外声明的包等,需要更改模板文件。%

%
\end{document}