\documentclass{article}
\usepackage{lastpage}
\usepackage{ragged2e}
\usepackage{pifont}
\usepackage{amsmath}

\usepackage{amsmath}%提供数学公式支持

\usepackage{graphics}%用于添加图片
\usepackage{graphicx}%加强插图命令
\newcommand{\figpath}[1]{contents/fig/#1}

\usepackage{fontspec}%用于配置字体
\usepackage[table]{xcolor}%用于各种颜色环境
\usepackage{enumitem}%用于定制list和enum
\usepackage{float}%用于控制Float环境,添加H参数(强制放在Here)
\usepackage[colorlinks,linkcolor=airforceblue,urlcolor=blue,anchorcolor=blue,citecolor=green]{hyperref}%用于超链接,另外添加该包目录会自动添加引用。

\usepackage[most]{tcolorbox}%用于添加各种边框支持
\usepackage[cache=true,outputdir=./out]{minted}%如果不保留临时文件就设置cache=false,如果输出设置了其他目录那么outputdir参数也有手动指定,否则会报错。
\tcbuselibrary{minted}%加载tcolorbox的代码风格

\usepackage[a4paper,left=3cm,right=3cm,top=3cm,bottom=3cm]{geometry}%用于控制版式
\usepackage{appendix}%用于控制附加文件
\usepackage{ifthen}

\usepackage{pdfpages}%用于支持插入其他pdf页
\usepackage{booktabs}%目前用于给表格添加 \toprule \midrule 等命令
\usepackage{marginnote} %用于边注
\usepackage[pagestyles,toctitles]{titlesec} %用于标题格式DIY
% \usepackage{fancyhdr}%用于排版页眉页脚
\usepackage{ragged2e} % 用于对齐
\usepackage{fixltx2e} %用于文本环境的下标
\usepackage{ulem} %用于好看的下划线、波浪线等修饰
\usepackage{pifont} %数学符号
\usepackage{amssymb} %数学符号

\usepackage{fontspec}
\setmainfont{DejaVu Serif}


\definecolor{langback}{RGB}{245,244,250}
\definecolor{langbacktitle}{RGB}{235,233,245}
\definecolor{langtitle}{RGB}{177,177,177}
\definecolor{langno}{RGB}{202,202,202}
\tcbset{arc=1mm}
\renewcommand{\theFancyVerbLine}{\sffamily\textcolor{langno}{\scriptsize\oldstylenums{\arabic{FancyVerbLine}}}}%重定义行号的格式
\newtcblisting{langbox}[1][tex]{%参考自https://reishin.me/tmux/ 的代码框样式
    arc=1mm,
    colframe=langbacktitle,
    colbacktitle=langbacktitle,
    coltitle=langtitle,
    fonttitle=\bfseries\sffamily,
    lefttitle=1mm,toptitle=0.5mm,bottomtitle=0.5mm,
    title = Code,
    drop shadow,
    listing engine=minted,
    minted style=colorful,
    minted language=#1,
    minted options={fontsize=\small,breaklines,autogobble,linenos,numbersep=2mm,xleftmargin=1mm},
    colback=langback,listing only,
    bottomrule=0mm,leftrule=0mm,toprule=0mm,rightrule=0mm,
    enhanced,
    % overlay={\begin{tcbclipinterior}\fill[langback] (frame.south west)rectangle ([xshift=5mm]frame.north west);\end{tcbclipinterior}}
}

\definecolor{boxback}{RGB}{245,246,250}
\newtcolorbox{markquote}{
    colback=boxback,fonttitle=\sffamily\bfseries,arc=0pt,
    boxrule=0pt,bottomrule=-1pt,toprule=-1pt,leftrule=-1pt,rightrule=-1pt,
    drop shadow,enhanced
}

\usepackage[UTF8,fontset=windowsnew,heading=true]{ctex}
\ctexset{
	section = {
	number = 第\chinese{section}章,
	format = \zihao{3}\bfseries,
	},
	subsection = {
	number = \arabic{section}.\arabic{subsection},
	format = \Large\bfseries
	},
	subsubsection = {
	number = \arabic{section}.\arabic{subsection}.\arabic{subsubsection},
	format = \Large\bfseries,
	},
    paragraph = {
	format = \large\bfseries,
	},
    subparagraph = {
	format = \large\bfseries,
	},
}

\setlength{\parindent}{2em}%设置首行缩进
\linespread{1.3}%设置行距

\setlength{\parskip}{0.5em}%设置段间距
\setcounter{tocdepth}{4}%设置目录级数
\setcounter{secnumdepth}{3}


\newtcbox{\inlang}[1][red]{on line,
arc=0pt,outer arc=0pt,colback=#1!10!white,colframe=#1!50!black,
boxsep=0pt,left=1pt,right=1pt,top=2pt,bottom=2pt,
boxrule=0pt,bottomrule=-1pt,toprule=-1pt,leftrule=-1pt,rightrule=-1pt}

\newlength\tablewidth


\definecolor{tablelinegray}{RGB}{221,221,221}
\definecolor{tablerowgray}{RGB}{247,247,247}
\definecolor{tabletopgray}{RGB}{245,246,250}
\definecolor{airforceblue}{rgb}{0.36, 0.54, 0.66}


\title{MarkTex特性说明}
\author{sailist}

\begin{document}
\normalsize
\maketitle
\tableofcontents
\newpage









\section{特性\textsubscript{下标在这里}}


\begin{itemize}
\item
支持目前主流的所有markdown语法(目前,脚注和xml标签暂时不支持)
\item
额外添加了下划线语法(\underline{下划线})
\item
表格自动调整列宽
\item
复选框支持三种
\item
无论是本地图片还是网络图片,都能够支持。
\end{itemize}



\section{效果演示}




本文用于演示和测试转换后的效果


\subsection{普通文本}


支持一般的文本和\textbf{加粗},\textit{斜体},\inlang{\small{行内代码}},和 $InLine Formula$ ,\href{http://github.com}{超链接},注意公式暂时不支持中文。


\sout{删除线},\underline{下划线}


\subsection{二级标题}




\subsubsection{三级标题}


目录编号支持到三级标题,可以通过修改latex文件或者直接更改模板来完成。


\paragraph{四级标题}


\subparagraph{五级标题}




\subsection{脚注}




可以支持脚注格式\footnote{这里是脚注的内容}






\subsection{表格}


支持一般的文本格式,暂时不支持表格内图片。另外,表格取消了浮动(float),因此不支持对表格的描述(caption),不过在Markdown中也没有对表格的描述,因此也不算功能不完善。


\begin{center}
\setlength\tablewidth{\dimexpr (\textwidth -4\tabcolsep)}
\arrayrulecolor{tablelinegray!75}
\rowcolors{2}{tablerowgray}{white}
\begin{tabular}{|p{0.500\tablewidth}<{\centering}|p{0.500\tablewidth}<{\centering}|}
\hline
\rowcolor{tabletopgray}
\textbf{ColA}&\textbf{ ColB }\\
\hline
 \textbf{Table Bold} &  \textit{Table Italic}\\
\hline
 \inlang{\small{Table Code}} &   $Table Formula$ \\
\hline
\href{http:///www.github.com}{Table line}&Table Text\\
\hline
\end{tabular}
\end{center}



\begin{center}
\setlength\tablewidth{\dimexpr (\textwidth -8\tabcolsep)}
\arrayrulecolor{tablelinegray!75}
\rowcolors{2}{tablerowgray}{white}
\begin{tabular}{|p{0.077\tablewidth}<{\centering}|p{0.077\tablewidth}<{\centering}|p{0.077\tablewidth}<{\centering}|p{0.769\tablewidth}<{\centering}|}
\hline
\rowcolor{tabletopgray}
\textbf{A}&\textbf{B}&\textbf{C}&\textbf{Long Text Sample Long Text Sample Long Text Sample Long Text Sample Long Text Sample Long Text Sample }\\
\hline
A&B&C&D\\
\hline
A&B&C&D\\
\hline
A&B&C&D\\
\hline
\end{tabular}
\end{center}



\subsection{列表和序号/itemize\&enumerate}


\begin{itemize}
\item
支持\textbf{加粗},\textit{斜体},\inlang{\small{行内代码}}, $Inline Formula$ ,\href{http:///www.github.com}{超链接}
\item
支持\textbf{加粗},\textit{斜体},\inlang{\small{行内代码}}, $Inline Formula$ ,\href{http:///www.github.com}{超链接}
\item
支持\textbf{加粗},\textit{斜体},\inlang{\small{行内代码}}, $Inline Formula$ ,\href{http:///www.github.com}{超链接}
\end{itemize}



\begin{enumerate}
\item
支持\textbf{加粗},\textit{斜体},\inlang{\small{行内代码}}, $Inline Formula$ ,\href{http:///www.github.com}{超链接}
\item
支持\textbf{加粗},\textit{斜体},\inlang{\small{行内代码}}, $Inline Formula$ ,\href{http:///www.github.com}{超链接}
\item
支持\textbf{加粗},\textit{斜体},\inlang{\small{行内代码}}, $Inline Formula$ ,\href{http:///www.github.com}{超链接}
\end{enumerate}



\begin{itemize}
\item[\rlap{\raisebox{0.3ex}{\hspace{0.4ex}\tiny \ding{52}}}$\square$]
支持
\item[\rlap{\raisebox{0.3ex}{\hspace{0.4ex}\scriptsize \ding{56}}}$\square$]
三种
\item[$\square$]
复选框格式
\end{itemize}



\subsection{图片}


和表格一样,取消了浮动,因此暂时不支持对图片的描述。不过本项目支持网络图片,会在转换的时候自动下载到本地,同时如果是非JPG或者PNG格式的图片,会转换为PNG格式。


\subsubsection{行内图片}




最新版本添加了行内图片,如果没有换行,那么该图片会被人为是行内图片,会自动调整高度适应一行:\raisebox{-0.5mm}{\includegraphics[height=1em]{images/1c59f8ef2aa3c5e527a22b7c258489d6.png}}


测试2:\raisebox{-0.5mm}{\includegraphics[height=1em]{images/23a8048a54a8f32cd7c0573a1e52f239.png}}图片之后


\subsubsection{行间图片}




\begin{center}
\vspace{\baselineskip}\includegraphics[width=0.8\textwidth]{images/1c59f8ef2aa3c5e527a22b7c258489d6.png}\vspace{\baselineskip}
\end{center}


相对路径:
\begin{center}
\vspace{\baselineskip}\includegraphics[width=0.8\textwidth]{images/ef84f157872e22d7cffcae03a00ea530.png}\vspace{\baselineskip}
\end{center}


\subsection{公式}


公式不支持中文,并且没有编号


\[
\\ f(x_i)=ax_i+b\\ \\
\]





\section{Tex文件插入}
可以通过使用include标签插入tex原文件,不过注意,不需要添加文档区,文档类等声明,插入的方式为完全将原文件复制粘贴到相应位置。

也因此,这种方式下如果使用了额外声明的包等,需要更改模板文件。


\subsection{符号支持}


符号的直接转换是比较方便的,做一个映射即可,但是符号可以存在于很多地方,甚至包括公式中,此时mathjax是可以识别的,但是latex不可以,这就导致了很多问题,一开始是做了一个折中,就是需要用户自己手动更改,但还是很麻烦,于是在\href{https://tex.stackexchange.com/questions/69901/how-to-typeset-greek-letters}{stackoverflow}上找到了解决方案,通过添加一个字体集的方式直接支持这些符号,目前支持的符号列举如下(可能支持更多符号,但没有经过测试):


\subsubsection{希腊字母}


αβγδεζηθικλμνξοπρστυφχψω


ΑΒΓΔΕΖΗΘΙΚΛΜΝΞΟΠΡΣΤΥΦΧΨΩ


\subsubsection{运算符号}


±×÷∣∤


⋅∘∗⊙⊕


≤≥≠≈≡


∑∏∐∈∉⊂⊃⊆⊇⊄


∧∨∩∪∃∀∇


⊥∠


∞∘′


∫∬∭


↑↓←→↔↕


\subsection{代码}


代码使用tcolorbox和minted,基本支持所有主流语言。支持的所有语言请参考 \href{https://www.overleaf.com/learn/latex/Code_Highlighting_with_minted}{Code Highlighting with minted}




\begin{langbox}[Python]
if __name__ == "__main__":
    print("hello world!")
\end{langbox}



\begin{langbox}[C++]
#include<stdio.h>
int main(){
    printf("hello world")
    return 0;
}

\end{langbox}



\subsection{引用}


\begin{markquote}

引用内环境和普通文本基本一致,但是不支持标题。
演示\textbf{加粗},\textit{斜体},\inlang{\small{行内代码}}, $Inline Formula$ ,\href{http:///www.github.com}{超链接}




\begin{itemize}
\item
支持\textbf{加粗},\textit{斜体},\inlang{\small{行内代码}}, $Inline Formula$ ,\href{http:///www.github.com}{超链接}
\end{itemize}


\begin{enumerate}
\item
支持\textbf{加粗},\textit{斜体},\inlang{\small{行内代码}}, $Inline Formula$ ,\href{http:///www.github.com}{超链接}
\end{enumerate}







\end{markquote}



\begin{markquote}

表格:




\begin{center}
\setlength\tablewidth{\dimexpr (\textwidth -4\tabcolsep)}
\arrayrulecolor{tablelinegray!75}
\rowcolors{2}{tablerowgray}{white}
\begin{tabular}{|p{0.500\tablewidth}<{\centering}|p{0.500\tablewidth}<{\centering}|}
\hline
\rowcolor{tabletopgray}
\textbf{ColA}&\textbf{ ColB }\\
\hline
 \textbf{Table Bold} &  \textit{Table Italic}\\
\hline
 \inlang{\small{Table Code}} &   $Table Formula$ \\
\hline
\href{http:///www.github.com}{Table line}&Table Text\\
\hline
\end{tabular}
\end{center}



公式:




\[
F(x_i) = wx_i+b\\
\]



图片:
\begin{center}
\vspace{\baselineskip}\includegraphics[width=0.8\textwidth]{images/1c59f8ef2aa3c5e527a22b7c258489d6.png}\vspace{\baselineskip}
\end{center}






\end{markquote}



\section{新特性{-}引入其他Markdown文档}




非常酷的特性!可以使用特殊的html标签来引入其他的MarkDown!


\normalsize



\subsection{简单表格}


\begin{center}
\setlength\tablewidth{\dimexpr (\textwidth -4\tabcolsep)}
\arrayrulecolor{tablelinegray!75}
\rowcolors{2}{tablerowgray}{white}
\begin{tabular}{|p{0.500\tablewidth}<{\centering}|p{0.500\tablewidth}<{\centering}|}
\hline
\rowcolor{tabletopgray}
\textbf{a}&\textbf{b}\\
\hline
c&d\\
\hline
\end{tabular}
\end{center}



\subsection{普通表格}


\begin{center}
\setlength\tablewidth{\dimexpr (\textwidth -4\tabcolsep)}
\arrayrulecolor{tablelinegray!75}
\rowcolors{2}{tablerowgray}{white}
\begin{tabular}{|p{0.500\tablewidth}<{\centering}|p{0.500\tablewidth}<{\centering}|}
\hline
\rowcolor{tabletopgray}
\textbf{sample text }&\textbf{ sample text}\\
\hline
c&d\\
\hline
\end{tabular}
\end{center}


\normalsize



\[
f(x_i)=ax_i+b\\
\]



\begin{markquote}
\[
F(x_i) = wx_i+b\\
\]


\end{markquote}



\end{document}